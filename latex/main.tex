\documentclass[a4paper]{article}
\usepackage[utf8]{inputenc}
\usepackage[slovene]{babel}
\usepackage{graphicx}
\usepackage{hyperref}
\usepackage[nottoc]{tocbibind}
\usepackage{minted}
\usepackage{listings}
\usepackage{caption}
\usepackage{subcaption}
\usepackage{amsmath}
\usepackage{ dsfont }
\usepackage{siunitx}
\usepackage{multimedia}
\usepackage[table,xcdraw]{xcolor}
\setlength\parindent{0pt}

\definecolor{codegreen}{rgb}{0,0.6,0}
\definecolor{codegray}{rgb}{0.5,0.5,0.5}
\definecolor{codepurple}{rgb}{0.58,0,0.82}
\definecolor{backcolour}{rgb}{0.95,0.95,0.92}
\newcommand{\ddd}{\mathrm{d}}
\newcommand\myworries[1]{\textcolor{red}{#1}}
\newcommand{\Dd}[3][{}]{\frac{\ddd^{#1} #2}{\ddd #3^{#1}}}

\lstdefinestyle{mystyle}{
    backgroundcolor=\color{backcolour},   
    commentstyle=\color{codegreen},
    keywordstyle=\color{magenta},
    numberstyle=\tiny\color{codegray},
    stringstyle=\color{codepurple},
    basicstyle=\ttfamily\footnotesize,
    breakatwhitespace=false,         
    breaklines=true,                 
    captionpos=b,                    
    keepspaces=true,                 
    numbers=left,                    
    numbersep=5pt,                  
    showspaces=false,                
    showstringspaces=false,
    showtabs=false,                  
    tabsize=2
}

\lstset{style=mystyle}

\begin{document}
\begin{titlepage}
    \begin{center}
        \includegraphics[]{logo.png}
        \vspace*{3cm}
        
        \Huge
        \textbf{Strojno učenje (machine learning)}
        
        \vspace{0.5cm}
        \large
        12. naloga pri Matematično-fizikalnem praktikumu

        \vspace{4.5cm}
        
        \textbf{Avtor:} Marko Urbanč (28191096)\ \\
        \textbf{Predavatelj:} prof. dr. Borut Paul Kerševan\ \\
        
        \vspace{2.8cm}
        
        \large
        8.9.2023
    \end{center}
\end{titlepage}
\tableofcontents
\newpage
\section{Uvod}
Medtem ko se je še nedavno zdelo strojno učenje prava temna magija (vsaj meni), je pravzaprav dandanes
uporaba različnih algoritmov strojnega učenja popolnoma vsakdanja in že rutinska. Ljudje se pravzaprav 
stalno srečujemo z različnimi algoritmi strojnega učenja, ki nam pomagajo pri različnih stvareh, kot so
npr. priporočila na Netflixu, Googlovi iskalni algoritmi, različni algoritmi za prepoznavanje objektov na
slikah, itd. Vse to so algoritmi, ki so zasnovani na strojnem učenju. \\

Prvotno sem želel pustiti prejšnji stavek brez pojasnila, vendar sem se odločil, da bom vseeno napisal nekaj 
besed o kar mislim. Poglejmo prvo Netflix. Netflix je spletna storitev za ogled filmov in serij. Ker je na Netflixu 
ogromno filmov in serij, je težko najti tisto, kar bi si želeli ogledati. Zato Netflix uporablja algoritem, ki na 
podlagi vaših prejšnjih ogledov in ocen priporoča filme in serije, ki bi vam lahko bili všeč 
\cite{lamkhede2021recommendations}. Ampak ne samo to, glede na to iz katere naprave gledate, ob kakšnem času dneva in 
drugih parametrih, ki jim Netflix pravi Kontekst \cite{Steck_Baltrunas_Elahi_Liang_Raimond_Basilico_2021}, pravzaprav 
so pa to \textit{contextual bandits}, ki se uporabljajo v enem okusu strojnega učenja. Več o tem kasneje. V kolikor 
sem uspel razumeti ta članek zgleda, da še več kot to, želijo praktično \textit{in real time} prilagajati priporočila, 
glede na to kakšen je vaš t.i. \textit{intent} \cite{10.1145/2959100.2959174}. Oh in vsi, ki si med seboj delite 
Netflix račune, nikakor ne skrbite, tudi to vas zna nekoč tepst kakšen algoritem strojnega 
učenja \cite{esmaeilzadeh2022abuse}. \\

Nadalje, Google. Google je spletni iskalnik, ki ga praktično vsi uporabljamo. Google uporablja algoritme strojnega
že precej dolgo časa. Leta 2015 so predstavili deep learning sistem RankBrain, ki je bil namenjen izboljšanju
rezultatov iskanja. RankBrain je bil namenjen predvsem iskanju poizvedb, ki jih Google še nikoli ni videl.
RankBrain je bil zasnovan tako, da je na podlagi preteklih iskanj in rezultatov iskanj, ki so jih uporabniki
izbrali, izboljševal rezultate iskanja. Od 2018 naprej so v Google Search v rabi nevronske mreže, ki so namenjene
predvsem izboljšanju rezultatov iskanja. Od 2019 naprej pa Google uporablja tudi BERT \cite{47751}, ki je bil ogromen
korak naprej v razumevanju naravnega jezika. \\

Na hitro o tipih strojnega učenja oz. osnovnih vrstah algoritmov strojnega učenja. Poznamo

\begin{itemize}
    \item Nadzorovano učenje (supervised learning)
    \begin{itemize}
        \item Klasifikacija (classification): Sortiranje podatkov v razrede. Primer: Razpoznavanje številk na sliki.
        \item Regresija (regression): Modeliranje odvisnosti med podatki. Primer: Napovedovanje cene nepremičnine.
    \end{itemize}
    \item Nenadzorovano učenje (unsupervised learning)
    \item Stimulirano učenje (reinforcement learning)
\end{itemize}

Poglejmo si zdaj uporabo strojnega učenja v fizikalnem kontekstu. V fiziki se strojno učenje uporablja za različne
namene. Največkrat uporabljamo algoritme prvega tipa, torej jih nadzorovano učimo. V fiziki visokih energij se strojno
učenje uporablja za razpoznavanje delcev v detektorjih, za razpoznavanje različnih pojavov, za izboljšanje različnih 
meritev, itd. To je nekaj kar bomo tudi sami počeli v tej nalogi. Iskali bomo Higgsov bozon v podatkih, ki jih je
zbrala ATLAS kolaboracija. \\

\subsection{Kako delujejo algoritmi strojnega učenja?}
Pred tem, pa še osnovno o tem, kako sploh delujejo algoritmi strojnega učenja. Imamo nabor parametrov $\mathcal{D} = 
\{(x_k, y_k)\}_{k=1}^N$


\section{Naloga}

\section{Opis reševanja}

\section{Rezultati}


\section{Komentarji in izboljšave}


\newpage
\bibliographystyle{unsrt}
\bibliography{sources}
\end{document}
